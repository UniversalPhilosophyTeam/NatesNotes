\documentclass[11pt]{article}
\usepackage{geometry}                
\geometry{letterpaper}                 
\usepackage[parfill]{parskip}        
\usepackage{graphicx}
\usepackage{subfigure}
\usepackage{amssymb}
\usepackage{amssymb}
\usepackage{amsmath}
\usepackage{epstopdf}
\usepackage{verbatim}
\usepackage{float}
\usepackage{grffile}
\usepackage{fullpage}
\usepackage{enumerate}
\usepackage{hyperref}
\usepackage[utf8]{inputenc}
\usepackage{gensymb}
\usepackage[T1]{fontenc}
\usepackage[hang,small]{caption}
\DeclareGraphicsRule{.tif}{png}{.png}{`convert #1 `dirname #1`/`basename #1 .tif`.png}

\graphicspath{ {C:/Users/Nate/Documents/School/EECS351/Nates Discussion Material/Discussion 1} }
\usepackage{listings}
\usepackage{color}
\usepackage{textcomp}
\definecolor{listinggray}{gray}{0.9}
\definecolor{lbcolor}{rgb}{1,1,1}
\lstset{
	backgroundcolor=\color{lbcolor},
	tabsize=4,
	rulecolor=,
	language=matlab,
	basicstyle= \scriptsize,
	upquote=true,
	aboveskip={1.5\baselineskip},
	columns=fixed,
        	showstringspaces=false,
        	extendedchars=true,
        	breaklines=true,
        	prebreak = \raisebox{0ex}[0ex][0ex]{\ensuremath{\hookleftarrow}},
        	frame=single,
        	showtabs=false,
        	showspaces=false,
        	showstringspaces=false,
        	identifierstyle=\ttfamily,
        	keywordstyle=\color[rgb]{0,0,1},
        	commentstyle=\color[rgb]{0.133,0.545,0.133},
        	stringstyle=\color[rgb]{0.627,0.126,0.941},
}


\begin{document}

\section*{EECS351 Discussion 1 with MATLAB demo SOLUTIONS, 09/08/16}
Nate Sawicki

\section*{6 \hspace{3mm} MATLAB Problem Solutions}

\subsection*{6.1 \hspace{3mm} Generate a Signal}
\begin{center}

>> x = [3 5 1];

\end{center}

\subsection*{6.2 \hspace{3mm} Plot a Signal}
\begin{center}

>> stem(x)

\end{center}

\subsection*{6.3 \hspace{3mm} Generate a longer signal}

This question was somewhat open ended. But to create a cosine wave, there's a certain way we often declare t:
\begin{center}

>> t = 0:1/(fs):duration;


\end{center}

\begin{center}


>> cosWave = cos(t);

\end{center}

\subsection*{6.4 \hspace{3mm} Music Processing Basics}

What is x[n]?
\begin{center}

$x[n] = \begin{cases}x(n/(fs)), & n \hspace{1mm} \epsilon \hspace{1mm} \mathbb{Z}$ (T is the sampling frequency) $\\ 0, & otherwise \end{cases}$    


\end{center}

\vspace{4mm}

\begin{center}

$x[n] = \begin{cases}cos(2\pi 400 n/44100), & n \hspace{1mm} \epsilon \hspace{1mm} \mathbb{Z}$ (fs is the sampling frequency) $\\ 0, & otherwise \end{cases}$    


\end{center}


\vspace{4mm}

Create x[n] in MATLAB using the variable t from 1.3. Name the variable x.

\begin{center}

>> t = 0:1/44100:duration;

\end{center}

\begin{center}

>> x = cos(2*pi*400*t)

\end{center}

\begin{center}

>> plot(t,x) or stem(t,x)

\end{center}

\begin{center}

>> xlabel('time')

\end{center}

\begin{center}

>> ylabel('Value of Cosine')

\end{center}

\subsection*{6.5 \hspace{3mm} Analyze the Signal}

\begin{center}

>> Fourier = fft(x)

\end{center}

\begin{center}

>> plot(abs(fftshift(Fourier)))

\end{center}

\section*{Theoretical Problems}

\subsection*{7 \hspace{3mm} Alternative Expressions for Sequences}
Let $x[n] = \begin{cases} \left(\frac{1}{2}\right)^n, & n \text{ nonnegative multiple of 4} \\ -\left(\frac{1}{2}\right)^n, & n \text{ nonnegative multiple of 2, but not a nonnegative multiple of 4} \\ 0, & \text{otherwise}\end{cases}$ \\
Express $x[n]$ mathematically in three different ways.

{\color{blue}
1. $x[n] = \{\underline{1}, 0, \frac{-1}{4}, 0, \frac{1}{16}, 0, \frac{-1}{64},... \}$ \\
\\
2. $x[n] = \delta[n]-\frac{1}{4}\delta[n-2]+\frac{1}{16}\delta[n-4]-\frac{1}{64}\delta[n]+...$ \\ \\
3. $x[n] = \sum\limits_{k=0}^\infty (-1)^k(\frac{1}{4})^k\delta[b-2k]$ \\ \\
4. $x[n] = u[n]cos\left(\frac{\pi}{2}n\right)\left(\frac{1}{2}\right)^n$ }







\end{document}