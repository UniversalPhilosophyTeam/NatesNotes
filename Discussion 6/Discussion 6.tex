\documentclass[11pt]{article}
\usepackage{geometry}                
\geometry{letterpaper}                 
\usepackage[parfill]{parskip}        
\usepackage{graphicx}
\usepackage{subfigure}
\usepackage{amssymb}
\usepackage{amssymb}
\usepackage{amsmath}
\usepackage{epstopdf}
\usepackage{verbatim}
\usepackage{float}
\usepackage{grffile}
\usepackage{fullpage}
\usepackage{enumerate}
\usepackage{amsmath}
\usepackage{hyperref}
\usepackage[utf8]{inputenc}
\usepackage{gensymb}
\usepackage[T1]{fontenc}
\usepackage[hang,small]{caption}
\DeclareGraphicsRule{.tif}{png}{.png}{`convert #1 `dirname #1`/`basename #1 .tif`.png}

\graphicspath{ {C:/Users/Nate/Documents/School/EECS351/Nates Discussion Material/Discussion 4} }
\usepackage{listings}
\usepackage{color}
\usepackage{textcomp}
\definecolor{listinggray}{gray}{0.9}
\definecolor{lbcolor}{rgb}{1,1,1}
\lstset{
	backgroundcolor=\color{lbcolor},
	tabsize=4,
	rulecolor=,
	language=matlab,
	basicstyle= \scriptsize,
	upquote=true,
	aboveskip={1.5\baselineskip},
	columns=fixed,
        	showstringspaces=false,
        	extendedchars=true,
        	breaklines=true,
        	prebreak = \raisebox{0ex}[0ex][0ex]{\ensuremath{\hookleftarrow}},
        	frame=single,
        	showtabs=false,
        	showspaces=false,
        	showstringspaces=false,
        	identifierstyle=\ttfamily,
        	keywordstyle=\color[rgb]{0,0,1},
        	commentstyle=\color[rgb]{0.133,0.545,0.133},
        	stringstyle=\color[rgb]{0.627,0.126,0.941},
}


\begin{document}

\section*{EECS351 Discussion 6 Problems, 10/13/16}
Nate Sawicki \newline
Select problems by Mai Le and Kevin Moon


\section{Convolution equations}

\vspace{3mm}

\begin{center}

$
x[n] * h[n] = \sum_{\tau  = -\infty}^{\infty} x[\tau] h[n - \tau]
$
\end{center}

\vspace{3mm}


\begin{center}

$
x[n] *  h[n] \longleftrightarrow X(\omega)H(\omega)
$
\end{center}

Note:\newline
(a) x[n] *  h[n] is a function of n, not $\tau$.\newline
(b) For EVERY index of n, you need to compute a potentially infinite sum.

\vspace{3mm}

\section{Graphical Intuition}

Understanding the graphical intuition behind its definition is one way to compute convolution. Here is my step by step process for computing convolution by looking at the problem graphically:\newline

\begin{center}

$
x[n] * h[n] = y[n]
$
\end{center}

\vspace{1.5mm}

(1) Pick either x[n] or h[n] (doesn't matter which) and call this $x_1$[n]. Draw this in solid ink, because you won't be changing it.\newline
(2) Pick the other signal and call it $x_2$[n]. Draw $x_2$[n] on separate axes directly underneath $x_1$[n]. You will be shifting $x_2$[n] frequently, so don't draw it too permanently.\newline
(3) Flip $x_2$[n] about the y-axis, call this $x_3$.\newline
(4) For EVERY value of n, compute the infinite sum associated with convolution for a fixed n. This infinite sum is computed by adding the element-wise product of $x_1$ and the shifted version of $x_3$ at each value of tau. This value of the infinite sum is equal to y[n] for a fixed n. A value of n = 3, corresponds to shifting $x_3$[n] to the right by 3.  A value of n = -5, corresponds to shifting $x_3$[n] to the left by 5. 

\section{Convolution example}

Let x[n] = [1, -1]\newline
Let h[n] = [3, 5, 1]

As a class, use the steps for graphical convolution to verify:

\begin{center}

x[n] * h[n] = [3, 2, -4, -1]

\end{center}


\section{Basic Problem}

Let $x_1$ have length N, let $x_2$ have length M. \newline

What is the length of $x_3$[n] = $x_1$[n] * $x_2$[n]


\section{Faster Method for Convolution}

There is a faster method for computing convolution, but it requires less understanding. This is what we call the "stack and sum" method:\newline

\begin{center}

$
x[n] * h[n] = y[n]
$
\end{center}

\begin{center}

$
y[n] = h[0]x[n]
$
$$
+ [0 \hspace{1.5mm} h[1]x[n]]
$$
$$
+ [0 \hspace{1.5mm} 0 \hspace{1.5mm} h[2]x[n]] + ...
$$
$$
+ [0...(N zeros) \hspace{1.5mm}  \hspace{1.5mm} h[N-1]x[n]]
$$


\end{center}


\vspace{1.5mm}





\section{Stack and Sum Method}

Let x[n] = [3, 5, 1]\newline
Let h[n] = [2, 1, 6]\newline

\vspace{3mm}
(a) Find x[n] * h[n] using the stack and sum method\newline
(b) Find h[n] * x[n] using the stack and sum method\newline
(c) Is part (a) the same as part (b)? What is the length of x[n] * h[n], does your answer agree with Problem 4?


\newpage

\section{Challenge Problem: Convolution}

In analog signal processing, it was often easier to think about multiplication in the frequency domain, rather than convolution in the time domain. Here's a problem where it is probably easier to think in terms of convolution.

Let:


\begin{center}


x[n] = [0 1 1 1]

\end{center}
\begin{center}


h[n] = [... , -1 ,2, -1, 2, -1 , ...] with h[0] = -1

\end{center}

\vspace{4mm}

(a) Find y[n] = x[n] * h[n]\newline
(b) Express X($\omega$) as a sum of three complex exponentials. Express H(w) as a sum of infinity complex exponentials.\newline
(c) Find Y($\omega$) and take the inverse DTFT to obtain y[n], you should get the same answer as part a


\section{Challenge Problem: Proofs}

(a) Show that x[n] * h[n] = h[n] * x[n]\newline
(b) Show the convolution property of the DTFT:



\begin{center}

$ 
x[n] * h[n] \longleftrightarrow X(\omega) H(\omega)
$
\end{center}




\end{document}