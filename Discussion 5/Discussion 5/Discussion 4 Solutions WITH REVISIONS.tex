\documentclass[11pt]{article}
\usepackage{geometry}                
\geometry{letterpaper}                 
\usepackage[parfill]{parskip}        
\usepackage{graphicx}
\usepackage{subfigure}
\usepackage{amssymb}
\usepackage{amssymb}
\usepackage{amsmath}
\usepackage{epstopdf}
\usepackage{verbatim}
\usepackage{float}
\usepackage{grffile}
\usepackage{fullpage}
\usepackage{enumerate}
\usepackage{hyperref}
\usepackage[utf8]{inputenc}
\usepackage{gensymb}
\usepackage[T1]{fontenc}
\usepackage[hang,small]{caption}
\DeclareGraphicsRule{.tif}{png}{.png}{`convert #1 `dirname #1`/`basename #1 .tif`.png}

\graphicspath{ {C:/Users/Nate/Documents/School/EECS351/Nates Discussion Material/Discussion 4} }
\usepackage{listings}
\usepackage{color}
\usepackage{textcomp}
\definecolor{listinggray}{gray}{0.9}
\definecolor{lbcolor}{rgb}{1,1,1}
\lstset{
	backgroundcolor=\color{lbcolor},
	tabsize=4,
	rulecolor=,
	language=matlab,
	basicstyle= \scriptsize,
	upquote=true,
	aboveskip={1.5\baselineskip},
	columns=fixed,
        	showstringspaces=false,
        	extendedchars=true,
        	breaklines=true,
        	prebreak = \raisebox{0ex}[0ex][0ex]{\ensuremath{\hookleftarrow}},
        	frame=single,
        	showtabs=false,
        	showspaces=false,
        	showstringspaces=false,
        	identifierstyle=\ttfamily,
        	keywordstyle=\color[rgb]{0,0,1},
        	commentstyle=\color[rgb]{0.133,0.545,0.133},
        	stringstyle=\color[rgb]{0.627,0.126,0.941},
}


\begin{document}

\section*{EECS351 Discussion 4 Problems, 09/26/16}
Nate Sawicki and Alex Ying \newline
Select problems by Mai Le and Kevin Moon


\section{Discrete-domain Basis Sequences}
Let $x[n] = \{\underline{1},1,2,3,5\}$. Represent $x[n]$ in each of the following bases. 

Note: You won't be asked to do anything like this on the homework or exams, but hopefully it will help you solidify your understanding of bases from lecture.

\subsection*{Unit Step Sequences}
Recall $u[n] = \begin{cases}1, & n \geq 0\\ 0, & n < 0 \end{cases}$. Let $\mathcal{S}_u = \{u[n-n_0]|n_0 \in \mathbb{Z}\}$ (unit steps shifted by any integer $n_0$) be your basis sequences. Represent $x[n]$ in $\mathcal{S}_u$.

In other words, show you can write $x[n]=\sum\limits_{k=-\infty}^\infty c_k u[n-k]$ by finding the values for $c_k$.

\subsection*{Three-Tap Rectangles}
Let $r[n] = \{\underline{1},1,1\}$ and $\mathcal{S}_r = \{r[n-n_0]|n_0 \in \mathbb{Z}\}$. Represent $x[n]$ in $\mathcal{S}_r$.

\subsection*{Three-Tap Triangle}
Let $t[n] = \{\underline{1},2,1\}$ and $\mathcal{S}_t = \{t[n-n_0]|n_0 \in \mathbb{Z}\}$. Represent $x[n]$ in $\mathcal{S}_t$.








\section*{2 Problems}
\subsection*{Problem 1}
Show that B is orthonormal.

\begin{center}
\[
B = 
\begin{bmatrix}
   3/\sqrt{10} \hspace{3mm}   1/\sqrt{10} \\
   1/\sqrt{10} \hspace{3mm}   -3/\sqrt{10}           
\end{bmatrix}
\]
\end{center}

\vspace{4mm}
\emph{Solution:}
Lets show that v1 and v2 can express the standard/canonical basis vectors using a weighted sum:

\vspace{5mm}
That is, find a particular  $\alpha$ and $\beta$ such that
\begin{center}
 $\alpha$$v_1$ + $\beta$$v_2$ = [1 0]$^{T}$
\end{center}
And find a particular  $c$ and $d$ such that
\begin{center}
 $c$$v_1$ + $d$$v_2$ = [0 1]$^{T}$
\end{center}
\begin{center}
\[
B = 
\begin{bmatrix}
   v_1  \hspace{3mm}  v_2 
\end{bmatrix}
\]
\end{center}

\vspace{4mm}

Consider ($\sqrt{10}v_1$ - 3$\sqrt{10}v_2$)/10 =


\begin{center}
\[
B = 
\begin{bmatrix}
   0 \\
 1
\end{bmatrix}
\]
\end{center}


\vspace{4mm}

Consider (3$\sqrt{10}v_1$ + $\sqrt{10}v_2$)/10 =

\begin{center}
\[
B = 
\begin{bmatrix}
   1 \\
 0
\end{bmatrix}
\]
\end{center}

\vspace{5mm}

\subsection*{2.2 \hspace{3mm} Useful Property}
Suppose B is a valid orthogonal basis matrix

\vspace{3mm}
Then
\begin{center}

 (B$^{-1}$) exists

\end{center}
and
\begin{center}

 (B$^{-1}$) = B$^{T}$

\end{center}



\vspace{5mm}

\subsection*{2.3}
Last week we spent a lot of time discussing what constitutes a basis. Suppose B is any valid basis, then any signal \textbf{x} $\epsilon$ $\mathbb{R}^{2}$ (length 2) can be expressed:

\vspace{3mm}
\begin{center}

B$\textbf{c}$ = \textbf{x}

\end{center}
\vspace{3mm}
\begin{center}
Where B is a valid basis for $\mathbb{R}^{2}$
\end{center}
\begin{center}
\textbf{c} = [$c_1$ $c_2$]$^{T}$ is called \emph{the coordinates}
\end{center}

\begin{center}
\[
B = 
\begin{bmatrix}
   b_1  \hspace{3mm}   b_3 \\
   b_2 \hspace{3mm}   b_4            
\end{bmatrix}
\]
\end{center}

Let $v_1$ = [$b_1$ $b_2$]$^T$, $v_2$ = [$b_3$ $b_4$]$^T$

\vspace{4mm}
\begin{center}
B =  [\textbf{$v_1$} \textbf{$v_2$}] 

\end{center}

\hspace{3mm}

Graphically, $v_1$ and $v_2$ represent coordinate axes.  The coordinates tell you how many $v_1$'s and $v_2$'s are needed to express \textbf{x}. So if $c_1$ = 2 and $c_2$ = 3, then you need 2$v_1$ and 3$v_2$ to create \textbf{x}. Symbolically:
\begin{center}

2$v_1$ + 3$v_2$ = \textbf{x}

\end{center}

\newpage

\section*{2.4 \hspace{3mm} Change of Basis}

Represent the signal \textbf{x} = [2 1]$^{T}$ the following ways: \newline
(a) using the standard/canonical basis\newline
(b) using the basis B from Problem 1, find the coordinates with respect to the new basis which represents the signal \textbf{x} = [2 1]$^{T}$\newline
(c) Draw the graphical representation of (a) and (b) in the same $\mathbb{R}^2$ plane. Verify that even though the coordinates for (a) and (b) are different, they still represent the same vector.

\vspace{5mm}
\emph{Solution:}\newline
(a)


\begin{center}

B$\textbf{c}$ = \textbf{x}

\end{center}
\begin{center}

\[
=
\begin{bmatrix}
   1 \hspace{3mm} 0 \\
 0 \hspace{3mm} 1
\end{bmatrix}
\begin{bmatrix}
   c_1\\
 c_2 
\end{bmatrix} 
= 
\begin{bmatrix}
   2\\
 1 
\end{bmatrix} 
\]

\end{center}

\vspace{3mm}
Thus $c_1$ = 2, and $c_2$ = 1. We just found the coordinates with respect to the canonical basis.\newline
(b)
\begin{center}

B$\textbf{c}$ = \textbf{x}

\end{center}
\begin{center}

\[
=
\begin{bmatrix}
   3/\sqrt{10} \hspace{3mm}   1/\sqrt{10} \\
   1/\sqrt{10} \hspace{3mm}   -3/\sqrt{10}           
\end{bmatrix}
\begin{bmatrix}
   c_1\\
 c_2 
\end{bmatrix} 
= 
\begin{bmatrix}
   2\\
 1 
\end{bmatrix} 
\]

\end{center}

Using the Property from 2:

\begin{center}

B$\textbf{c}$ = \textbf{x}

\end{center}

\begin{center}

$\textbf{c}$ = $B^{-1}$ \textbf{x}

\end{center}
\begin{center}

$\textbf{c}$ = $B^{T}$ \textbf{x}

\end{center}

\begin{center}

$\textbf{c}$ = [2.21 -.32]$^T$

\end{center}



\vspace{3mm}
This $c_1$ = 2.21, and $c_2$ = -.32. We just found the coordinates with respect to the basis B.
\newpage

\subsection*{(c)}







\vspace{5mm}



\subsection*{3 Inner Product}
\begin{center}

||z||$^2$

\end{center}
\begin{center}

= ||x + y||$^2$

\end{center}
\begin{center}

= (x+y)(x+y)

\end{center}

\begin{center}

= $x^2 + 2xy + y^2$

\end{center}

\begin{center}

xy = 0 since x $\perp$ y

\end{center}

\begin{center}

= $x^2 + y^2$ = ||x||$^2$ + ||y||$^2$

\end{center}



\end{document}