\documentclass[11pt]{article}
\usepackage{geometry}                
\geometry{letterpaper}                 
\usepackage[parfill]{parskip}        
\usepackage{graphicx}
\usepackage{subfigure}
\usepackage{amssymb}
\usepackage{amssymb}
\usepackage{amsmath}
\usepackage{epstopdf}
\usepackage{verbatim}
\usepackage{float}
\usepackage{grffile}
\usepackage{fullpage}
\usepackage{enumerate}
\usepackage{amsmath}
\usepackage{hyperref}
\usepackage[utf8]{inputenc}
\usepackage{gensymb}
\usepackage[T1]{fontenc}
\usepackage[hang,small]{caption}
\DeclareGraphicsRule{.tif}{png}{.png}{`convert #1 `dirname #1`/`basename #1 .tif`.png}

\graphicspath{ {C:/Users/Nate/Documents/School/EECS351/Nates Discussion Material/Discussion 4} }
\usepackage{listings}
\usepackage{color}
\usepackage{textcomp}
\definecolor{listinggray}{gray}{0.9}
\definecolor{lbcolor}{rgb}{1,1,1}
\lstset{
	backgroundcolor=\color{lbcolor},
	tabsize=4,
	rulecolor=,
	language=matlab,
	basicstyle= \scriptsize,
	upquote=true,
	aboveskip={1.5\baselineskip},
	columns=fixed,
        	showstringspaces=false,
        	extendedchars=true,
        	breaklines=true,
        	prebreak = \raisebox{0ex}[0ex][0ex]{\ensuremath{\hookleftarrow}},
        	frame=single,
        	showtabs=false,
        	showspaces=false,
        	showstringspaces=false,
        	identifierstyle=\ttfamily,
        	keywordstyle=\color[rgb]{0,0,1},
        	commentstyle=\color[rgb]{0.133,0.545,0.133},
        	stringstyle=\color[rgb]{0.627,0.126,0.941},
}


\begin{document}

\section*{EECS351 Discussion 5 Problems, 10/13/16}
Nate Sawicki \newline
Select problems by Mai Le and Kevin Moon

\section{Digital Frequency vs. Real Frequency}

\begin{center}
$f_{real}$ = $\frac{\omega_{digital}}{2\pi T_s}$

\end{center}

The highest digital frequency represented is $\pi$ thus 

\begin{center}

$f_{max}$ = $F_s$/2
\end{center}

The maximum frequency represented by a 44,100 sampling rate is 22,050. This is easy to remember because humans hear a frequency spectrum of approximately 20-20,000 Hz.

\begin{center}

>>N = length(x)\\

>>frequencies = -Fs/2:Fs/N:Fs/2-Fs/N        \hspace{5mm}     // final -Fs/N term ensures length N\\
>>plot(frequencies,fftshift(abs(fft(x))))

\end{center}




\section{Discrete-Time Fourier Transform DTFT equations}

\vspace{3mm}

\begin{center}

$
X(\omega) = X_{2 \pi} (w) = X(e^{j \omega}) = \sum_{n = -\infty}^{\infty} x[n]  e^{-j \omega n}
$
\end{center}

\vspace{5mm}

\begin{center}

$
x[n] = (1/{2 \pi}) \int_{2 \pi}  X(\omega) e^{j \omega n} d \omega
$
\end{center}

\vspace{5mm}


\section{Periodicity of DTFT}

The DTFT is often written the following way, to denote that any DTFT is periodic with period $2 \pi$: \newline

\begin{center}

$
X_{2 \pi} (w) =\sum_{n = -\infty}^{\infty} x[n]  e^{-j \omega n}
$
\end{center}

\vspace{4mm}

Show that the DTFT is periodic with period $2 \pi$.



\vspace{3mm}


\section{Linearity of the DTFT}

The following property is called the Linearity Property of the DTFT:

\begin{center}

$$
\alpha x[n] + \beta y[n] \longleftrightarrow  \alpha X(\omega) + \beta Y(\omega)
$$
\end{center}

\vspace{4mm}

Proof: (always start with the original equation !!!!!!)\newline

\begin{center}

$ 
Z(e^{j \omega}) = \sum_{n = -\infty}^{\infty} z[n]  e^{-j \omega n}
$
\end{center}

\vspace{3mm}

Let z[n] = $\alpha x[n] + \beta y[n]$ \newline

\begin{center}

$ 
Z(e^{j \omega}) = \sum_{n = -\infty}^{\infty} ( \alpha x[n] + \beta y[n])  e^{-j \omega n}
$
\end{center}

\newpage

\begin{center}

$ 
Z(e^{j \omega}) = \sum_{n = -\infty}^{\infty} ( \alpha x[n] )  e^{-j \omega n} + \sum_{n = -\infty}^{\infty} (  \beta y[n])  e^{-j \omega n}
$
\end{center}

\vspace{3mm}

\begin{center}

$ 
Z(e^{j \omega}) = \alpha \sum_{n = -\infty}^{\infty} (  x[n] )  e^{-j \omega n} + \beta \sum_{n = -\infty}^{\infty} (  y[n])  e^{-j \omega n}
$
\end{center}

\vspace{3mm}

\begin{center}

$ 
Z(e^{j \omega}) = \alpha X(\omega) + \beta Y(\omega)
$
\end{center}


\section{Shift in Time Property of DTFT}

The following property is used to find the DTFT of a signal that's been shifted: \newline

\begin{center}

$
x[n - \alpha]  \longleftrightarrow  X(\omega) e^{- j \alpha \omega}
$
\end{center}

\vspace{4mm}

Using the definition of DTFT, show that the Shift in Time Property is true. 


\section{Modulation Property of DTFT (shift in frequency)}

The following property is used to find the DTFT of a signal that's been shifted: \newline

\begin{center}

$
e^{j \alpha n} x[n]  \longleftrightarrow  X(\omega - \alpha)
$
\end{center}

\vspace{4mm}

Using the definition of DTFT, show that the Modulation Property is true.

\vspace{5mm}

\section{Summary of (Some) DTFT Properties}

\vspace{3mm}


\begin{center}
$
\alpha x[n] + \beta y[n] \longleftrightarrow  \alpha X(\omega) + \beta Y(\omega)
$

\end{center}
\begin{center}
$
x[n - \alpha]  \longleftrightarrow  X(\omega) e^{- j \alpha \omega}
$

\end{center}


\begin{center}

$
e^{j \alpha n} x[n]  \longleftrightarrow  X(\omega - \alpha)
$
\end{center}

\begin{center}

$
x[kn]  \longleftrightarrow  X(\omega / k)
$
\end{center}
\begin{center}

$
x[-n]  \longleftrightarrow  X(-\omega)
$
\end{center}
\begin{center}

$
x[n]^{*} \longleftrightarrow  X(-\omega)^{*}
$
\end{center}
\begin{center}

$
x[-n]^{*} \longleftrightarrow  X(\omega)^{*}
$
\end{center}

\begin{center}

$
-(j n) x[n] \longleftrightarrow (d/d\omega) X(\omega)
$
\end{center}

\begin{center}

$
x[n] * y[n] \longleftrightarrow X(\omega) Y(\omega)
$
\end{center}



\newpage

\section{Extra Problems: Cross Correlation}

Recall the definition of convoluion:
\vspace{1mm}
\begin{center}
$
a[n] * b[n] = z[n] =  \sum_{\tau = -\infty}^{\infty} a[\tau] b[n - \tau]
$

\end{center}

\vspace{4mm}

There is an expression called the cross-correlation property of the DTFT:
\vspace{3mm}
\begin{center}
$
a^{*}[-n] * b[n] \longleftrightarrow \hspace{1.5mm} ?
$

\end{center}

\vspace{5mm}

(a) Find the DTFT of $a^{*}[-n] * b[n]$ using DTFT properties from part 6.\newline
(b) Prove your answer from part (a) without using DTFT properties.



\section{Extra Problems: Zero Padding 1}

Let $x_1$[n] = [$x_0$ $x_1$ $x_2$ ... $x_{N-1}$]$^{T}$, a signal of length N.\newline

Now consider $x_2$, which is $x_1$ followed by N zeros: \newline
Let $x_2$[n] = [$x_0$ $x_1$ $x_2$ ... $x_{N-1}$ 0 0 0 ... 0]$^{T}$, a signal of length 2N.\newline

You might be tempted to think we can gain some kind of information about the DTFT by adding zeros to the end of this signal. Find X$_2$($\omega$) in terms of X$_1$($\omega$) by expanding the DTFT equation one term at a time.

\vspace{4mm}

\section{Extra Problems: Zero Padding 2}

Let $x_1$[n] = [$x_0$ $x_1$ $x_2$ ... $x_{N-1}$]$^{T}$, a signal of length N.\newline

Now consider $x_3$, which is $x_1$ with a zero inserted between each index: \newline
Let $x_3$[n] = [$x_0$ 0 $x_1$ 0 $x_2$ ... 0 $x_{N-1}$ 0 ]$^{T}$, a signal of length 2N.\newline

You might be tempted to think we can gain some kind of information about the DTFT by inserting zeros into this signal. Find X$_3$($\omega$) in terms of X$_1$($\omega$) by expanding the DTFT equation one term at a time.




\end{document}