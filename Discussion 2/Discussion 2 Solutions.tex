\documentclass[11pt]{article}
\usepackage{geometry}                
\geometry{letterpaper}                 
\usepackage[parfill]{parskip}        
\usepackage{graphicx}
\usepackage{subfigure}
\usepackage{amssymb}
\usepackage{amssymb}
\usepackage{amsmath}
\usepackage{epstopdf}
\usepackage{verbatim}
\usepackage{float}
\usepackage{grffile}
\usepackage{fullpage}
\usepackage{enumerate}
\usepackage{hyperref}
\usepackage[utf8]{inputenc}
\usepackage{gensymb}
\usepackage[T1]{fontenc}
\usepackage[hang,small]{caption}
\DeclareGraphicsRule{.tif}{png}{.png}{`convert #1 `dirname #1`/`basename #1 .tif`.png}

\graphicspath{ {C:/Users/Nate/Documents/School/EECS351/Nates Discussion Material/Discussion 1} }
\usepackage{listings}
\usepackage{color}
\usepackage{textcomp}
\definecolor{listinggray}{gray}{0.9}
\definecolor{lbcolor}{rgb}{1,1,1}
\lstset{
	backgroundcolor=\color{lbcolor},
	tabsize=4,
	rulecolor=,
	language=matlab,
	basicstyle= \scriptsize,
	upquote=true,
	aboveskip={1.5\baselineskip},
	columns=fixed,
        	showstringspaces=false,
        	extendedchars=true,
        	breaklines=true,
        	prebreak = \raisebox{0ex}[0ex][0ex]{\ensuremath{\hookleftarrow}},
        	frame=single,
        	showtabs=false,
        	showspaces=false,
        	showstringspaces=false,
        	identifierstyle=\ttfamily,
        	keywordstyle=\color[rgb]{0,0,1},
        	commentstyle=\color[rgb]{0.133,0.545,0.133},
        	stringstyle=\color[rgb]{0.627,0.126,0.941},
}


\begin{document}

\section*{EECS351 Discussion 1 with MATLAB demo SOLUTIONS, 09/08/16}
Nate Sawicki

\section*{4 \hspace{3mm} Basis Examples}

\subsection*{4.1 \hspace{3mm} Basis Problem}

Consider the Hilbert Space $\mathbb{R}^3$ with real scalars. Find a basis for $\mathbb{R}^3$  that includes the following two vectors:
\[
\begin{bmatrix}
   1        \\
    -2      \\
    1      
\end{bmatrix}
,
\begin{bmatrix}
    1  \\
    4 \\
    -2 
\end{bmatrix}
\]
Solution:\newline
We try the following vector as the third basis vector
\begin{center}
\[
\begin{bmatrix}
   -2        \\
    -2      \\
    2     
\end{bmatrix}
\]


\end{center}

\vspace{3mm}
We can verify these vectors as a valid basis if the matrix

\begin{center}
\[
\begin{bmatrix}
   1  \hspace{3mm} 1 \hspace{ 3mm} -2      \\
   -2  \hspace{3mm} 4 \hspace{ 3mm} -2  \\
    1  \hspace{3mm} -2 \hspace{ 3mm} 2   
\end{bmatrix}
\]
\end{center}

can be row-reduced to the identity matrix. This is equivalent to writing the canonical basis as a linear combination of basis vectors. You can also plug the above matrix into MATLAB. The function rank(matrix) returns the number of linearly independent columns. 

\subsection*{4.2 \hspace{3mm} Show vectors form a basis}
Show that these four vectors form a basis for $\mathbb{R}^4$:
\[
\begin{bmatrix}
   1        \\
    0      \\
   0       \\
    0      
\end{bmatrix}
,
\begin{bmatrix}
      1        \\
    1      \\
   0       \\
    0      
\end{bmatrix}
,
\begin{bmatrix}
   1        \\
    1     \\
   1    \\
    0      
\end{bmatrix}
,
\begin{bmatrix}
       1        \\
    1      \\
   1       \\
    1      
\end{bmatrix}
\]

\vspace{3mm}
Solution:\newline
Similar to 4.1, it is easy to show that these vectors row reduce to the identity matrix, or canonical basis. Thus, these vectors represent a valid basis for $\mathbb{R}^4$

\subsection*{4.3 \hspace{3mm} Orthonormal Basis}
Similar to 4.1 and 4.2, these vectors can be row-reduced to the identity matrix, so we know these form a valid basis for $\mathbb{R}^4$. To satisfy orthonormality, we just need to verify that each vector satisfies the following properties:

\begin{center}

$v_i$$^T$$v_j$$^T$ = 0 when i $\neq$ j.

||$v_i$|| = $\sqrt{v_i^T v_i}$ = 1 for all i 

\end{center}

\subsection*{4.4 \hspace{3mm} Finding vectors that DON'T form a Basis}
Any value of $\alpha$ will make this a valid basis except for $\alpha$ = 1. Note, $\alpha$ = -1 makes these vectors orthogonal.


\subsection*{4.5 \hspace{3mm} Functions as a basis}

Consider the Hilbert space of continuous-time real-valued polynomials of degree 3 defined on the interval [a,b]. Show that the polynomials\newline

\begin{center}

$x_0(t) = 1, x_1(t) = t, x_2(t) = t^2, x_3(t) = t^3$

\end{center}

\vspace{3mm}

form a basis for this space.\newline\newline

Solution: For this to be a basis, we require linear independece between the vectors. This means that we need to show that uniformily \emph{for all} t$\epsilon$[a,b], the only $B_i$ that satisfy the equation

\begin{center}

$\beta_0 x_0(t) + \beta_1 x_1(t) + \beta_2 x_2(t) + \beta_3 x_3(t) = 0   \hspace{3mm}$  (for all t over the interval)

\end{center}

are $B_i$ = 0 for all i.\newline

We can do this by choosing four distinct values of t and then showing that the resulting system of equations has only one solution with $B_i$ = 0 for all i. Feel free to pick four values and verify the solution for $B_i$, but these $x_i$ do in fact represent a basis.

\section*{Representing the same signal with different bases}

\section*{5.2 \hspace{3mm} Standard Basis}
{\color{blue}
$c = \{\underline{1}, 1, 2, 3, 5\}$
}


\section*{5.2 \hspace{3mm} Unit Step Sequences}
Recall $u[n] = \begin{cases}1, & n \geq 0\\ 0, & n < 0 \end{cases}$. Let $\mathcal{S}_u = \{u[n-n_0]|n_0 \in \mathbb{Z}\}$ (unit steps shifted by any integer $n_0$) be your basis sequences. Represent $x[n]$ in $\mathcal{S}_u$.

In other words, show you can write $x[n]=\sum\limits_{k=-\infty}^\infty c_k u[n-k]$ by finding the values for $c_k$.

{\color{blue}
$c = \{\underline{1}, 0, 1, 1, 2, -5\}$
}

\subsection*{5.3 \hspace{3mm} Three-Tap Rectangles}
Let $r[n] = \{\underline{1},1,1\}$ and $\mathcal{S}_r = \{r[n-n_0]|n_0 \in \mathbb{Z}\}$. Represent $x[n]$ in $\mathcal{S}_r$.

{\color{blue}
$c = \{\underline{1}, 0, 1, 2, 0, 2, -4, 2, 2, -4, 2, 2, -4, \ldots \}$
}


\section*{Vector Space, Hilbert Spaces, and Inner Products}

\subsection*{6.1 \hspace{3mm} Inner Product}
Prove the Pythagorean theorem for norms defined by inner products. That is, if z = x + y and \newline
x $\perp$ y, then ||x||$^2$ + ||y||$^2$ = ||z||$^2$\newline

Solution:
\begin{center}

||z||$^2$

\end{center}
\begin{center}

= ||x + y||$^2$

\end{center}
\begin{center}

= (x+y)(x+y)

\end{center}

\begin{center}

= $x^2 + 2xy + y^2$

\end{center}

\begin{center}

xy = 0 since x $\perp$ y

\end{center}

\begin{center}

= $x^2 + y^2$ = ||x||$^2$ + ||y||$^2$

\end{center}

\subsection*{7.1 \hspace{3mm} Complex Space}

Nate needs a new solution to this one. Post your answers to Piazza :)


\end{document}