\documentclass[11pt]{article}
\usepackage{geometry}                
\geometry{letterpaper}                 
\usepackage[parfill]{parskip}        
\usepackage{graphicx}
\usepackage{subfigure}
\usepackage{amssymb}
\usepackage{amssymb}
\usepackage{amsmath}
\usepackage{epstopdf}
\usepackage{verbatim}
\usepackage{float}
\usepackage{grffile}
\usepackage{fullpage}
\usepackage{enumerate}
\usepackage{hyperref}
\usepackage[utf8]{inputenc}
\usepackage{gensymb}
\usepackage[T1]{fontenc}
\usepackage[hang,small]{caption}
\DeclareGraphicsRule{.tif}{png}{.png}{`convert #1 `dirname #1`/`basename #1 .tif`.png}

\graphicspath{ {C:/Users/Nate/Documents/School/EECS351/Nates Discussion Material/Discussion 2} }
\usepackage{listings}
\usepackage{color}
\usepackage{textcomp}
\definecolor{listinggray}{gray}{0.9}
\definecolor{lbcolor}{rgb}{1,1,1}
\lstset{
	backgroundcolor=\color{lbcolor},
	tabsize=4,
	rulecolor=,
	language=matlab,
	basicstyle= \scriptsize,
	upquote=true,
	aboveskip={1.5\baselineskip},
	columns=fixed,
        	showstringspaces=false,
        	extendedchars=true,
        	breaklines=true,
        	prebreak = \raisebox{0ex}[0ex][0ex]{\ensuremath{\hookleftarrow}},
        	frame=single,
        	showtabs=false,
        	showspaces=false,
        	showstringspaces=false,
        	identifierstyle=\ttfamily,
        	keywordstyle=\color[rgb]{0,0,1},
        	commentstyle=\color[rgb]{0.133,0.545,0.133},
        	stringstyle=\color[rgb]{0.627,0.126,0.941},
}


\begin{document}

\section*{EECS351 Discussion 2, 1/17/17}
Nate Sawicki \newline
Select problems by Mai Le and Kevin Moon

\section{Periodicity of mulitiple sinusoids}

Suppose x[n] is the summation of m sinusoids with different fundamental periods:

\vspace{3mm}

\begin{center}

x[n] = $cos[\frac{n\pi}{4}]$ + $sin[\frac{n\pi}{32}]$ + ...

\end{center}


\vspace{3mm}

Assuming the $i^{th}$ sinusoid has fundamental period $T_i$, the fundamental period of x[n] is the Least Common Multiple of ($T_1, T_2, ..., T_m$)\newline

\vspace{20mm}

\begin{center}
Let x[n]  = $cos[\frac{n\pi}{4}]$ + $sin[\frac{n\pi}{32}]$ 
\end{center}

\vspace{3mm}
$T_1$ = 8, $T_2$ = 64, thus the Fundamental Period of x[n] is LCM(8,64) = 64\newline

\vspace{3mm}

\section{What is a basis?}
A basis for a class of signals is a collection of M signals in the class that have the property that any other signal in that class can be written as a weighted sum of those signals.

\begin{center}
y[n] = $\sum\limits_{m = 1}^M a_m x_m[n]$
\end{center}

\vspace{4mm}

Example:
\begin{center}
\[
\begin{bmatrix}
   1        \\
    0      \\
   0          
\end{bmatrix}
,
\begin{bmatrix}
      0        \\
    1      \\
   0       \\    
\end{bmatrix}
,
\begin{bmatrix}
   0        \\
    0     \\
   1    \\
    
\end{bmatrix}
\]
\end{center}
 
can represent the following vector using a weighted sum

\begin{center}
\[
\begin{bmatrix}
   1        \\
    4      \\
   9          
\end{bmatrix}
\]
\end{center}

using the weights $a_1 = 1$, $a_2 = 4$, and $a_3 = 9$. In fact, the three vectors form a  basis for any signal of length three. These are called the standard basis or canonical basis vectors for $\mathbb{R}^3$.

\section{Strategies for identifying a basis}
One way to characterize a basis for signals of length N:

\vspace{3mm}

\begin{center}
y[n] = $\sum\limits_{m = 1}^{N-1} \beta_m y_m[n] = 0$ implies that $\beta_k$ = 0 for all m = 0, ..., N - 1.
\end{center}

\vspace{3mm}

If the condition above is true, then the vectors $y_1, ..., y_{N-1}$ are \emph{linearly independent}. If any combination of weighted sum of $y_m$ add up to 0, then the vectors $y_1, ..., y_{N-1}$ are \emph{linearly dependent}

\vspace{4mm}

Example:
\[
\begin{bmatrix}
   1  \hspace{3mm}  3 \hspace{3mm}   -4  \\
\hspace{1mm}   1 \hspace{3mm}    9  \hspace{3mm}  -10\\
    0 \hspace{3mm}     8 \hspace{3mm}  -8
\end{bmatrix}
\]


If the $i^{th}$ column represents the vector $y_i$, do these vectors represent a valid basis?\newline
NO! using the weights $\beta_1$ =  $\beta_2$  = $\beta_3$ = 1, we can see:

\begin{center}
y[n] = $\sum\limits_{m = 1}^{N-1} \beta_m y_m[n] = 0$
\end{center}


\section{Identifying a basis by forming the canonical basis}
A collection of signals consitute a basis for $\mathbb{R}^{n}$ if one can form the n canonical basis vectors as a weighted sum of the collection.

Example:\newline
Do the columns of A constitute a basis for $\mathbb{R}^{3}$?

\[
A = 
\begin{bmatrix}
   1  \hspace{3mm}  2\hspace{3mm}   0 \\
    0 \hspace{3mm}    1  \hspace{4mm}  0\\
    1 \hspace{3mm}     2\hspace{3mm}  1
\end{bmatrix}
\]

\newpage
We must show that we can form the 3 canonical basis vectors for $\mathbb{R}^{3}$ as a weighted sum of the columns of A. Recall the canonical basis vectors for  $\mathbb{R}^{3}$:

\begin{center}
\[
\begin{bmatrix}
   1        \\
    0      \\
   0          
\end{bmatrix}
,
\begin{bmatrix}
      0        \\
    1      \\
   0       \\    
\end{bmatrix}
,
\begin{bmatrix}
   0        \\
    0     \\
   1    \\
    
\end{bmatrix}
\]
\end{center}

\vspace{4mm}
Solution:
\[
\begin{bmatrix}
   1        \\
    0      \\
   0          
\end{bmatrix}
= 1*column1 + 0*column2 +(-1)*column3
\]

\[
\begin{bmatrix}
   0      \\
    1      \\
   0          
\end{bmatrix}
= (-2)*column1 + 1*column2 +0*column3
\]

\[
\begin{bmatrix}
   0      \\
    0      \\
   1          
\end{bmatrix}
= 0*column1 + 0*column2 +1*column3
\]



\section{Problems}

\subsection{Basis Problem}

Consider the Hilbert Space $\mathbb{R}^3$ with real scalars. Find a basis for $\mathbb{R}^3$  that includes the following two vectors:
\[
\begin{bmatrix}
   1        \\
    -2      \\
    1      
\end{bmatrix}
,
\begin{bmatrix}
    1  \\
    4 \\
    -2 
\end{bmatrix}
\]

\subsection{Show vectors form a basis}

Show that these four vectors form a basis for $\mathbb{R}^4$:
\[
\begin{bmatrix}
   1        \\
    0      \\
   0       \\
    0      
\end{bmatrix}
,
\begin{bmatrix}
      1        \\
    1      \\
   0       \\
    0      
\end{bmatrix}
,
\begin{bmatrix}
   1        \\
    1     \\
   1    \\
    0      
\end{bmatrix}
,
\begin{bmatrix}
       1        \\
    1      \\
   1       \\
    1      
\end{bmatrix}
\]

\subsection{Orthonormal Basis}

Show that these four vectors form an orthonormal basis for $\mathbb{R}^4$ (these are called the Bell basis):
\[
\begin{bmatrix}
   \frac{1}{\sqrt{2}} \\
    0      \\
   0       \\
    \frac{1}{\sqrt{2}}       
\end{bmatrix}
,
\begin{bmatrix}
      \frac{1}{\sqrt{2}}         \\
    0      \\
   0       \\
    -\frac{1}{\sqrt{2}}      
\end{bmatrix}
,
\begin{bmatrix}
   0     \\
    \frac{1}{\sqrt{2}}     \\
   \frac{1}{\sqrt{2}}  \\
    0      
\end{bmatrix}
,
\begin{bmatrix}
      0     \\
    \frac{1}{\sqrt{2}}     \\
-\frac{1}{\sqrt{2}}  \\
    0    
\end{bmatrix}
\]

\subsection{Finding vectors that DONT form a Basis}

What values of $\alpha$ make the following vectors an invalid basis?
\[
\begin{bmatrix}
   \alpha \\
    1      \\
   1       \\
    1   
\end{bmatrix}
,
\begin{bmatrix}
     1 \\
    \alpha   \\
   1       \\
    1   
\end{bmatrix}
,
\begin{bmatrix}
  1 \\
    1      \\
   \alpha    \\
    1   
\end{bmatrix}
,
\begin{bmatrix}
      1 \\
    1      \\
   1       \\
    \alpha  
\end{bmatrix}
\]



\subsection{Functions as a basis}
Consider the Hilbert space of continuous-time real-valued polynomials of degree 3 defined on the interval [a,b]. Show that the polynomials\newline

\begin{center}

$x_0(t) = 1, x_1(t) = t, x_2(t) = t^2, x_3(t) = t^3$

\end{center}

\vspace{3mm}

form a basis for this space. Is the orthonormal basis where we use the same inner product as $L_2$ and a = -1, b = -1?




\section{Representing the same signal with different bases}

Let x[n] =[\underline{1}, 1, 2, 3, 5]. Represent x[n] in each of the following bases.

\vspace{5mm}

\subsection{Standard Basis}

\vspace{1.5mm}

\subsection{Unit Step Sequence}
Recall $u[n] = \begin{cases}1, & n \geq 0\\ 0, & n < 0 \end{cases}$. Let $\mathcal{S}_u = \{u[n-n_0]|n_0 \in \mathbb{Z}\}$ (unit steps shifted by any integer $n_0$) be your basis sequences. Represent $x[n]$ in $\mathcal{S}_u$.

In other words, show you can write $x[n]=\sum\limits_{k=-\infty}^\infty c_k u[n-k]$ by finding the values for $c_k$.

\vspace{1.5mm}

\subsection{Three-Tap Rectangles Basis}
Let $r[n] = \{\underline{1},1,1\}$ and $\mathcal{S}_r = \{r[n-n_0]|n_0 \in \mathbb{Z}\}$. Represent $x[n]$ in $\mathcal{S}_r$.

\vspace{1.5mm}


\section{Vector Spaces, Hilbert Spaces, and Inner Products}

\subsection{Inner Products}
Prove the Pythagorean theorem for norms defined by inner products. That is, if z = x + y and \newline
x $\perp$ y, then ||x||$^2$ + ||y||$^2$ = ||z||$^2$






\end{document}