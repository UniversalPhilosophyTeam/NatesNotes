\documentclass[11pt]{article}
\usepackage{geometry}                
\geometry{letterpaper}                 
\usepackage[parfill]{parskip}        
\usepackage{graphicx}
\usepackage{subfigure}
\usepackage{amssymb}
\usepackage{amssymb}
\usepackage{amsmath}
\usepackage{epstopdf}
\usepackage{verbatim}
\usepackage{float}
\usepackage{grffile}
\usepackage{fullpage}
\usepackage{enumerate}
\usepackage{hyperref}
\usepackage[utf8]{inputenc}
\usepackage{gensymb}
\usepackage[T1]{fontenc}
\usepackage[hang,small]{caption}
\DeclareGraphicsRule{.tif}{png}{.png}{`convert #1 `dirname #1`/`basename #1 .tif`.png}

\graphicspath{ {C:/Users/Nate/Documents/School/EECS351/Nates Discussion Material/Discussion 3} }
\usepackage{listings}
\usepackage{color}
\usepackage{textcomp}
\definecolor{listinggray}{gray}{0.9}
\definecolor{lbcolor}{rgb}{1,1,1}
\lstset{
	backgroundcolor=\color{lbcolor},
	tabsize=4,
	rulecolor=,
	language=matlab,
	basicstyle= \scriptsize,
	upquote=true,
	aboveskip={1.5\baselineskip},
	columns=fixed,
        	showstringspaces=false,
        	extendedchars=true,
        	breaklines=true,
        	prebreak = \raisebox{0ex}[0ex][0ex]{\ensuremath{\hookleftarrow}},
        	frame=single,
        	showtabs=false,
        	showspaces=false,
        	showstringspaces=false,
        	identifierstyle=\ttfamily,
        	keywordstyle=\color[rgb]{0,0,1},
        	commentstyle=\color[rgb]{0.133,0.545,0.133},
        	stringstyle=\color[rgb]{0.627,0.126,0.941},
}


\begin{document}

\section*{EECS351 Discussion 3, 1/23/2017}
Nate Sawicki \newline
Select problems by Mai Le and Kevin Moon

\section{Inner Product}
Prove the Pythagorean theorem for norms defined by inner products. That is, if z = x + y and \newline
x $\perp$ y, then ||x||$^2$ + ||y||$^2$ = ||z||$^2$

\section{Valid or Invalid Inner Product}
Consider signals in $\mathbb{R}^2$. Your friend tells you he has discovered a new inner product. He writes the expression for the new inner product below (length 2 signals).

\vspace{2mm}
\begin{center}
<x,y> = $x_1$$^3$$y_1$$^3$+$x_2$$y_2$
\end{center}
\vspace{2mm}

a)\newline
Does the proposed computation satisfy the positivity requirement of an inner product? \newline(<x,x> $\geq$ 0 and <x,x> = 0 iff x = 0)\newline

b)\newline
Recall that for an inner product space, the following property must be true. <$\alpha$x, y> = $\alpha$*<x,y>. For a real vector space,  only containing real entries, the following property must be satisfied.
\vspace{2mm}
\begin{center}
<$\alpha$x, y> = $\alpha$<x,y>
\end{center}
\vspace{2mm}
Has your friend satisfied this requirement?
\vspace{2mm}


\section{DFT Basis}

Recall the following valid un-normalized basis function.
\begin{center}
w$_k$[n] = W$_N^{-nk}$ = e$^{\frac{j2\pi nk}{N}}$
\end{center}

a) Generate the Fourier Basis Vectors for $\mathbb{R}^3$

\vspace{2mm}

b) Verify that the Vectors you found in a) are orthogonal


\newpage

c) Recall the definition of the DFT:

\begin{center}

X[k]= $\sum_n x[n] e^{-j2\pi nk/N}$

\end{center}

Compute X[k] for x[n] = [3 5 1]\newline

d) Compute the inverse DFT of X[k]\newline

\vspace{2mm}

e) Compute the coordinates of x[n] with respect to the normalized Basis from part a)\newline
\vspace{2mm}

f) Let $\gamma$ represent the normalizing constant you used in e) to create the normalized basis from a). Using $\gamma$ explain the relationship between X[k] and the coordinates you found in e)


\vspace{3mm}

\section{Valid Basis?}

Are the following set of vectors a valid basis for $\mathbb{R}^3$

\begin{center}

 
$
b_1 = 
\begin{bmatrix}

1 \\
0 \\
8 \\

\end{bmatrix}
$
\end{center}

\begin{center}

 
$
b_2 = 
\begin{bmatrix}

0 \\
1 \\
-3

\end{bmatrix}
$
\end{center}

\section{Valid Basis?}

Are the following set of vectors a valid basis for $\mathbb{R}^3$

\begin{center}

 
$
b_1 = 
\begin{bmatrix}

1 \\
0 \\
8 \\

\end{bmatrix}
$
\end{center}

\begin{center}

 
$
b_2 = 
\begin{bmatrix}

0 \\
1 \\
-8

\end{bmatrix}
$
\end{center}

\begin{center}

 
$
b_3 = 
\begin{bmatrix}

0 \\
1 \\
0

\end{bmatrix}
$
\end{center}




\vspace{5mm}

\begin{center}
\[
B = 
\begin{bmatrix}
  b_1 \hspace{2mm} b_2 \hspace{2mm} b_3           
\end{bmatrix}
\]
\end{center}

\newpage
\section{A different way of thinking}

Your friend has a favorite two digit number. But your friend is cheeky and he doesn't want to give you this important information without a bit of trickery. \newline
The secret number is encoded as a length two signal. He gives you the \textbf{coordinates with respect to the Normalized Basis Function from part 3} of the length two signal.

\begin{center}

$c_0$ = $\frac{9}{\sqrt{2}}$
\end{center}


\begin{center}

$c_1$ =  $\frac{7}{\sqrt{2}}$
\end{center}

What is your friend's favorite number?

\section{Free Response}

Express the length 4 signal x = [1 2 1 4]$^T$  as a weighted sum of $\frac{1}{N}$W$_N^{-nk}$ terms  (4 terms total)



\end{document}