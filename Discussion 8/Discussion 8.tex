\documentclass[11pt]{article}
\usepackage{geometry}                
\geometry{letterpaper}                 
\usepackage[parfill]{parskip}        
\usepackage{graphicx}
\usepackage{subfigure}
\usepackage{amssymb}
\usepackage{amssymb}
\usepackage{amsmath}
\usepackage{epstopdf}
\usepackage{verbatim}
\usepackage{float}
\usepackage{grffile}
\usepackage{fullpage}
\usepackage{enumerate}
\usepackage{amsmath}
\usepackage{hyperref}
\usepackage[utf8]{inputenc}
\usepackage{gensymb}
\usepackage[T1]{fontenc}
\usepackage[hang,small]{caption}
\DeclareGraphicsRule{.tif}{png}{.png}{`convert #1 `dirname #1`/`basename #1 .tif`.png}

\graphicspath{ {C:/Users/Nate/Documents/School/EECS351/Nates Discussion Material/Discussion 4} }
\usepackage{listings}
\usepackage{color}
\usepackage{textcomp}
\definecolor{listinggray}{gray}{0.9}
\definecolor{lbcolor}{rgb}{1,1,1}
\lstset{
	backgroundcolor=\color{lbcolor},
	tabsize=4,
	rulecolor=,
	language=matlab,
	basicstyle= \scriptsize,
	upquote=true,
	aboveskip={1.5\baselineskip},
	columns=fixed,
        	showstringspaces=false,
        	extendedchars=true,
        	breaklines=true,
        	prebreak = \raisebox{0ex}[0ex][0ex]{\ensuremath{\hookleftarrow}},
        	frame=single,
        	showtabs=false,
        	showspaces=false,
        	showstringspaces=false,
        	identifierstyle=\ttfamily,
        	keywordstyle=\color[rgb]{0,0,1},
        	commentstyle=\color[rgb]{0.133,0.545,0.133},
        	stringstyle=\color[rgb]{0.627,0.126,0.941},
}


\begin{document}

\section*{EECS351 Discussion 8 Problems, 11/10/16}
Nate Sawicki \newline
Select problems by Mai Le and Kevin Moon


\section{System Diagrams}

Consider the following Transfer Function:

\vspace{3mm}
\begin{center}

$
H(z) = 8 + \frac{-7+8z^{-1}}{1 - .75z^{-1} + .125z^{-2}}
$
\end{center}

\vspace{3mm}

Note:\newline
(a) Find the LCCDE (difference equation) y[n] in terms of shifted y[n] and shifted x[n]. \newline
(b) Draw the system diagram for the difference equation you found in part (a)

\vspace{3mm}

\section{Properties of ROC}
Consider a signal x[n], which has the following z-transform:
\vspace{2mm}

\begin{center}

$
X(z) = \frac{2z(z-\frac{1}{12})}{(z-\frac{1}{2})(z+\frac{1}{3})}
$
\end{center}
|z| > 1/2

\vspace{3mm}

(a) Determine the partial fraction expansion of X(z)\newline
(b) Sketch the ROC for X(z). Based on your sketch, does the DTFT for x[n] exist? Is x[n] right sided, left sided or neither? Is x[n] causal?\newline
(c) Find x[n] using the inverse z-transform. Based on your answer for x[n]: Is x[n] right sided, left sided or neither? Is x[n] causal?\newline

\section{Z-transform}

Determine the inverse z-transform for the following:

\begin{center}

\vspace{4mm}

$
X(z) = log( 1 + \alpha z^{-1})
$
\end{center}
|z| > |$\alpha$|

\vspace{3mm}

\begin{center}
HINT:  nx[n] $\longleftrightarrow -z \frac{dX(z)}{dz}$
\end{center}


\section{2D Basis}

Recall in 1D, the standard (canonical) basis for $\mathbb{R}^2$ :

\begin{center}
\[
\begin{bmatrix}
   1        \\
    0      
            
\end{bmatrix}
,
\begin{bmatrix}
      0        \\
    1      
\end{bmatrix}
\]
\end{center}

In 2D, the standard basis for $\mathbb{R}^{2x2}$ :

\begin{center}
\[
\begin{bmatrix}
   1  \hspace{2mm}    0  \\
    0   \hspace{2mm}  0 
            
\end{bmatrix}
,
\begin{bmatrix}
   0  \hspace{2mm}    1  \\
    0   \hspace{2mm}  0 
            
\end{bmatrix}
,
\begin{bmatrix}
   0 \hspace{2mm}    0  \\
    1   \hspace{2mm}  0 
            
\end{bmatrix}
,
\begin{bmatrix}
   0  \hspace{2mm}    0  \\
    0   \hspace{2mm}  1 
            
\end{bmatrix}
\]
\end{center}

Similarly to $\mathbb{R}^{2}$, we can show the standard basis is valid by showing:

\begin{center}
\[ \alpha_1
\begin{bmatrix}
   1  \hspace{2mm}    0  \\
    0   \hspace{2mm}  0 
            
\end{bmatrix}
+
\alpha_2
\begin{bmatrix}
   0  \hspace{2mm}    1  \\
    0   \hspace{2mm}  0 
            
\end{bmatrix}
+
\alpha_3
\begin{bmatrix}
   0 \hspace{2mm}    0  \\
    1   \hspace{2mm}  0 
            
\end{bmatrix}
+ 
\alpha_4
\begin{bmatrix}
   0  \hspace{2mm}    0  \\
    0   \hspace{2mm}  1 
            
\end{bmatrix}
=
\begin{bmatrix}
   0  \hspace{2mm}    0  \\
    0   \hspace{2mm}  0 
            
\end{bmatrix}
\]
\end{center}

Implies $\alpha_1 = \alpha_2 = \alpha_3 = \alpha_4 = 0$\newline
We could also show that a linear combination of basis matrices form the standard basis.

\vspace{4mm}

\section{2D Basis Question}

(a) Show

\begin{center}
\[ 
\begin{bmatrix}
   -1  \hspace{2mm}    -2  \\
    0   \hspace{2mm}  0 
            
\end{bmatrix}
,
\begin{bmatrix}
   2  \hspace{2mm}    3  \\
    4   \hspace{2mm}  5 
            
\end{bmatrix}
,
\begin{bmatrix}
   0 \hspace{2mm}    0  \\
    -3   \hspace{2mm}  -3 
            
\end{bmatrix}
, 
\begin{bmatrix}
   -1  \hspace{2mm}    -1  \\
    -1   \hspace{2mm}  -2
            
\end{bmatrix}
\]
\end{center}

\vspace{1.5mm}

is not a valid basis for $\mathbb{R}^{2x2}$

\newpage

\section*{2D Basis Question continued}

(b) Consider the following for matrices:

\begin{center}
\[ 
\begin{bmatrix}
   1  \hspace{2mm}    0  \\
    0   \hspace{2mm}  0 
            
\end{bmatrix}
,
\begin{bmatrix}
   1  \hspace{2mm}    2  \\
    1   \hspace{2mm}  1 
            
\end{bmatrix}
,
\begin{bmatrix}
   1 \hspace{2mm}    0  \\
    1   \hspace{2mm}  1
            
\end{bmatrix}
, 
\begin{bmatrix}
   0  \hspace{2mm}    0  \\
    0  \hspace{2mm}  1
            
\end{bmatrix}
\]
\end{center}

\vspace{1.5mm}

Show the canonical (standard) basis can be formed as a linear combination of the above matrices.


\vspace{4mm}

(c) Consider the following matrix:

\begin{center}
\[ 
\begin{bmatrix}
   1296 \hspace{2mm}    4899  \\
    3516   \hspace{2mm}  0135
            
\end{bmatrix}
\]
\end{center}

\vspace{1.5mm}

Can this matrix be expressed as a linear combination of the matrices from part (b)?



\end{document}